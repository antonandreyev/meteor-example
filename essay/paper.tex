\documentclass[a4paper,12pt]{article}

\usepackage{fontenc}
\usepackage{polyglossia}
\setdefaultlanguage{russian}
\setotherlanguage{english}
\defaultfontfeatures{Ligatures=TeX}

\setmainfont{CMU Serif}
\setsansfont{CMU Sans Serif}
\setmonofont{CMU Typewriter Text}
\newfontfamily\cyrillicfont[Script=Cyrillic]{CMU Serif}
\newfontfamily\cyrillicfontsf[Script=Cyrillic]{CMU Sans Serif}
\newfontfamily\cyrillicfonttt[Script=Cyrillic]{CMU Typewriter Text}

\tolerance=10000
\usepackage[a4paper,margin=2cm]{geometry}
\usepackage{setspace}
\usepackage{amsmath}
\usepackage{amssymb}
\usepackage{ifthen}
\usepackage{tikz}
\usepackage{indentfirst}
\usepackage{alltt}
\renewcommand{\baselinestretch}{1.5}

\begin{document}
\title{Реферат по JavaScript-фреймворку Meteor}
\author{А.~А.~Андреев, 22609}
\date{\today}
\maketitle

\large

\section*{Введение}
Трендом последних нескольких лет в Web-разработке являются приложения реального
времени и их разновидность --- реактивные приложения. Реактивность подразумевает
мгновенное обновление данных на клиентской стороне при обновлении их на сервере
(без необходимости обновлять страницу). Часто такие приложение выполняются в виде
одностраничных приложений, т.е. переходы между ресурсами приложения выполняются
без загрузки новой страницы, чаще всего с помощью асинхронной загрузки данных
с сервера. 

Закономерно, для облегчения создания таких приложений появился ряд программных
каркасов (фреймворков) для языка JavaScript, как основного языка клиентской
Web-Разработки. Некоторые из них предоставляют уровень абстракции
только для уровня представления (напрмер, ReactJS), другие предлагают
вариацию паттерна MVC (BackboneJS, AngularJS) и опирются на серверную сторону
через REST- или JSON-интерфейс.

Еще одна вариация фреймворк-архитектуры предлагается Meteor (MeteorJS) \cite{meteor}. 
Этот программный каркас предлагает full stack JavaScript разработку
с NodeJS на серверной стороне. В данном реферате рассматриваются 
подробности архитектуры фреймворка Meteor, основы и тонкости создания
приложений на его основе (на примере приложения для размещения новостей).

\section{Архитектура MeteorJS}
Meteor предлагает использование одного языка -- JavaScript -- и на
клиентской стороне (web-обозреватель, web-view на мобильных платформах), и
на серверной стороне (NodeJS). При этом все API (в том числе и к базе
данных) -- нативные для этого языка.

Для общения между двумя сторонами в Meteor используется собственный протокол --
Distributed Data Protocol (DDP). Он подразумевает общение между двумя сторонами
посредством JSON-объектов и пропагандирует модель публикация-подписка,
что позволяет асинхронно получать данные при обновлении их на сервере
без необходимости постоянного опроса сервера. В рамках протокола создается
постоянное соединение между клиентской и серверной стороной и сервер посылает
обновленные данные без инициации обмена клиентом.

Кроме того, Meteor поддерживает и стандартный для других фреймворков обмен данными
посредством HTTP-сообщений.

На серверной стороне Meteor, несмотря на занятие только одного порта,
работает сразу два web-сервера: сервер DDP (на основе SockJS и технологии
Web-сокетов), который обеспечивает реактивноевзаимодействие, и сервер HTTP (на
основе API NodeJS), который обеспечивает передачу статических файлов и
обработку классических HTTP-запросов.

В качестве СУБД Meteor предлагает использовать MongoDB, для чего имеется
собственный API, интегрированный с другими компонентами фреймворка (например,
протоколом DDP). Путем установки расширений Meteor позволяет организовать 
взаимодействие с другими СУБД, такими как PostgreSQL.

Фреймворком предоставляется возможность установки пакетов расширений
с помощью собственного менеджера пакетов Atmosphere.

\section{Основы разработки с помощью Meteor на примере новостного приложения}
Для обозначения основных моментов и тонкостей разработки приложений с помощью
Meteor в данном реферате будет описан процесс разработки приложения для размещения
Новостей. Приложение будет предоставлять возможность размещения новостей
(заголовок + текст) авторизованными пользователями и чтения новостей
всеми пользователями.

\subsection{Установка Meteor}

Дистрибутив фреймворка включает в себя не только файлы самого фреймворка, 
в него интегрированы NodeJS и MongoDB, что облегчает
первоначальную настройку приложения (однако усложняет установку приложений 
на production-сервер в плане гибкости настроек). После
установки дистрибутива, соответствующего целевой операционной системе,
из коммандной строки будет доступна команда ``\$~meteor~create~имя\_приложения'',
с помощью которой происходит инициализация начальной структуры нового приложения.

Рассматриваемый фреймворк имеет также собственный менеджер пакетов по названием
Atmosphere с соответствующим репозиторием пакетов (см. \cite{atmosphere}).
Установка пакетов осуществляется с помощью команды ``\$~meteor~add~имя\_пакета'',
а удалить с помощью ``\$~meteor~remove~имя\_пакета''.

\subsection{Файловая структура приложений}

После инициализации приложения будет создана следующая файловая структура:
\begin{itemize}
	\item package.json --- файл, описывающий приложение для npm;
	\item server/main.js --- файл, являющийся точкой входа для
	всего server-side кода;
	\item client/main.js --- файл, являющийся точкой входа для
	всего client-side кода;
	\item client/main.html --- файл с описанием представления приложения,
	включая шаблоны;
	\item client/main.css --- основной файл стилей приложения;
	\item .meteor --- каталог со служебными файлами фреймворка,
	включая список зависимостей, настройки приложения и т.п.;
	\item .meteor/local --- каталог со служебными локальными файлами 
	приложения, включая файлы базы данных, установленные пакеты 
	расширений, скомпилированные js-файлы.
\end{itemize}

В Meteor поощряется использование директив import/export из спецификации
ES2015. Так, создателяи фреймворка предлагается разделять приложение
на небольшие модули и импортировать их друг из друга. Исходя из
данных рекомендаций, в Новостном приложении будет использоваться следующая
дополнительная файловая структура:

\begin{itemize}
	\item imports --- каталог с модулями приложения;
	\item imports/startup/\{client и server\} --- код, который должен выполнится
	при старте приложения на клиентской и на серверной стороне соответственно;
	\item imports/api --- модули, описывающие предметную область приложения
	и вспомогательные функции;
	\item imorts/api/accounts-config.js --- инициализация и настройка системы аккаунтов пользователей;
	\item imports/ui --- модули уровня представления, включая шаблоны (.html) 
	и инициализирующий и вспомогательный код для шаблонов (.js).
\end{itemize}

Все соответствующие файлы из каталога ``startup'' (и
других, которые требуются) должны быть
импортированы в ``client/main.js'' и ``server/main.js''.

\subsection{Работа с базой данных}
В качестве СУБД по умолчанию Meteor использует MongoDB \cite{mongo}. 
Данные хранятся в виде коллекций документов произвольной структуры.
Доступ к БД осуществляется через объект ``Mongo'', который подключается
следующим образом:

\begin{alltt}
import \{ Mongo \} from 'meteor/mongo';
\end{alltt}

Документы БД в Meteor представлены, как объекты JavaScript.
Работа с коллекциями осуществляется следующим образом:

\begin{alltt}
// создание
var coll = new Mongo.Collection('collection\_name'); 
// получение всех документов
coll.find();
// запрос по документам
coll.find(\{someProperty: someVal\});
// получение одного найденного документа
coll.findOne(\{someProperty: someVal\});
// вставка документа
coll.insert(\{someProperty: someVal\});
// удаление всех удовлетворяющих элементов
coll.remove(\{someProperty: someVal\});
\end{alltt}

Подробнее см. общий API MongoDB. Заметим одну особенность Meteor:
коллекции по умолчанию доступны и на стороне сервера, и на стороне
клиента. При этом, при создании коллекции на стороне сервера
данная коллекция будет создана в БД, а при создании
коллекции на стороне клиента будет создано кеширующее подключение
к коллекции на сервере. На клиенте при этом по умолчанию будут
доступны все операции CRUD.

MongoDB является NoSQL базой данных и данные в ней по умолчанию
не валидируются. Для включения проверки данных в соответствии
с некоторой схемой данных можно использовать собственный (сложный!)
механизм MongoDB или использовать валидацию на стороне приложения
перед вставкой данных. Для Meteor для этого есть, например, пакет
aldeed:simple-schema.

Определение схемы для коллекции осуществляется следующим образом:
\begin{alltt}
	import \{ SimpleSchema \} from 'meteor/aldeed:simple-schema';
	
	coll.schema = new SimpleSchema(\{...\});
\end{alltt}

Включение автоматической валидации документов при операциях
insert и update можно осуществить следующим образом:

\begin{alltt}
	coll.attachSchema(coll.schema);
\end{alltt}

При работе с некорректными данными будет выбрашено ValidationError.

Для рассматриваемого новостного приложения спроектируем единственную
требуемую коллекцию --- коллекцию новостей. Meteor предлагает размещать
файлы с описанием коллекций в ``imports/api/''. В этом каталоге
создадим файл ``News.js'' с описанием коллекции новостей.
Схема будет следующая:

\begin{alltt}
News.schema = new SimpleSchema(\{
	// идентификатор новости
    \_id: \{ type: String, regEx: SimpleSchema.RegEx.Id \},
    // заголовок новости
    title: \{ type: String \},
    // основной новостной текст новости (в виде HTML)
    text: \{ type: String \},
    // дана создания новости
    date: \{ type: Date \},
    // имя пользователя, который создал новость
    username : \{ type: String \}
\});
\end{alltt}

Чтобы сделать коллекцию доступной в других модулях
приложения, в верхней части файла добавим:

\begin{alltt}
export const News = new Mongo.Collection('News');
\end{alltt}

\subsection{Роутинг}
Meteor ориентирован на создание одностраничных приложений, для которых
не нужен роутинг. Однако, для поддержки некоторых возможностей обработка
адресной строки и перенаправления между страницами все таки необходимо.
В нашем приложении предполагается существование отдельной страницы
для каждой новости и уникального адреса для них (чтобы доступ к
отдельной новости можно было бы легко получить извне, например
для индексации в поисковых системах).

Одним из доступных роутеров (которые распространяются в виде пакетов)
является ``kadira:flow-router''.

Для определения маршрута, который будет обрабатываться приложением
(на стороне клиента!), можно использовать следующую конструкцию:

\begin{alltt}
import \{ FlowRouter \} from 'meteor/kadira:flow-router';

FlowRouter.route('/some/path/:someVariable', \{
    name: 'SomeName.show',
    action(params, queryParams) \{
        // какие-либо действия
        console.log("some log");
    \}
\});
\end{alltt}

В данном случае, при совпадении запрашиваемого адреса с
определенным адресом будут выполнены действия, определенные
в методе ``action''. Параметры адресного запроса
указываются начиная с двоеточия. Объект ``params''
будет содержать имена и значения определенных параметров 
адресной строки, а объект ``queryParams'' ---
параметров GET запроса (если есть).

Для нашего приложения потребуются два маршрута.
Маршрут для корня приложения (адрес '/') и маршрут
для отдельной новости (адрес '/post/:\_id'). 

\begin{alltt}
FlowRouter.route('/post/:\_id', \{
    name: 'Post.show',
    action() \{  
        ...
    \}
\});

FlowRouter.route('/', \{
    name: 'main.show',
    action() \{  
        ...
    \}
\});
\end{alltt}

Предполагается, что файл с описанием маршрутов будет
размещен в ``imports/startup/client'', а модуль, описываемый
файлом, будет импортирован ``client/main.js''.

\subsection{Аутентификация и авторизация}
Для работы с учетными записями пользователей Meteor
предлагает использовать набор пакетов, среди которых
``accounts-ui'' для предоставления интерфейса для доступа
к аккаунту (например, форма аутентификации) и 
``accounts-password'' для авторизации с помощью пароля 
(альтернативой могут быть OAuth и данные внешних сервисов,
таких как Facebook).

Для настройки механизма авторизации необходимо предлагается
создать файл в ``imports/startup''. Файл нашего приложения будет
содержать только две инструкции: одна для подключения механизма
авторизации, другая для включения использования имени
пользователя вместо e-mail.

\begin{alltt}
import \{ Accounts \} from 'meteor/accounts-base';

Accounts.ui.config(\{
    passwordSignupFields: 'USERNAME\_ONLY',
\});
\end{alltt}

Модуль нужно импортировать из ``client/main.js''.

В приложении данные о пользователе будут доступны
в виде коллекции MongoDB Meteor.users, функций
``Meteor.userId()'' для получения идентификатора
текущего пользователя, ``Meteor.user()'' для
получения объекта, представляющего текущего пользователя.
Из представлений доступен объект ``currentUser''. 

\subsection{Шаблоны представлений}
Шаблоны представлений в Meteor требуют создания
двух файлов: HTML-файл с разметкой и логикой
шаблона (например, вывод элементов коллекции) и 
JS-файл с инициализацией шаблона и передачей
в него данных (почти controller).

Файлы шаблонов предполагается хранить в ``imports/ui'',
а в качестве шаблонизатора по умолчанию используется
``Handlebars''.

Для удобной работы в паре с роутером в рассматриваемом
новостном приложении используется пакет
``kadira:blaze-layout''. Он позволяет использовать
базовый шаблон для оформления страницы и расширять.

Модифицируем определение маршрутов следующим образом:

\begin{alltt}
import \{ BlazeLayout \} from 'meteor/kadira:blaze-layout';	
	
import '../../ui/layout.js';
import '../../ui/news.js';
import '../../ui/full\_post.js';
import '../../ui/post.js';	
	
FlowRouter.route('/post/:\_id', \{
    name: 'Post.show',
    action() \{  
        BlazeLayout.render('layout', \{ main: 'full\_post' \});
    \}
\});

FlowRouter.route('/', \{
    name: 'main.show',
    action() \{  
	    BlazeLayout.render('layout', \{ main: 'news' \});
    \}
\});
\end{alltt}

Теперь при совпадении маршрутов будет рендериться шаблон
``layout'', а в блоке ``main'' этого шабона будет 
рендериться шаблон, подходящий к маршруту.

Блок в шаблоне определяется следующим образом:
\begin{alltt}
\{\{> Template.dynamic template=main\}\}
\end{alltt}

Здесь ``Template'' импортируется из 'meteor/templating'
в JS-части шаблона.

Расширяющие шаблоны следующие: 
\begin{itemize}
	\item post описывает одну новость;
	\item news выводит список новостей, используя для каждого шаблон post;
	\item full\_post описывает новость в таком формате, в котором она должна быть
	выведена на индивидуальной странице новости;
\end{itemize}

В js-файлах необходимо импортировать шаблоны, которые предполагается переиспользовать.
Вообще, Meteor пропагандирует использование переиспользуемых компонентов, как, например,
компонент post в рассматриваемом новостном приложении.

\subsection{Формы и события}
Формы определяются в HTML-файлах шаблонов, а обработка форм
осуществляется по событиям, действия к которым привязываются
в стиле JQuery.

В нашем новостном приложении необходима форма для добавления новости,
которую мы расположим на странице новостей (шаблон news). Доступ
к этой форме будем предоставлять только авторизованным пользователям: 

\begin{alltt}
\{\{#if currentUser\}\}
  <hr/>
  <form class="add-news">
    <input type="text" name="title" placeholder="Заголовок"/>
    <textarea name="text" placeholder="Текст новости" ></textarea>
    <input type="submit" value="Добавить"/>
  </form>
\{\{/if\}\}
\end{alltt}

По нажатию кнопки ``Добавить'' возникнет событие ``submit''
при обработке которого мы должны добавить новую новость.
Обработчик для события мы назначем в ``/imports/ui/news.js'':

\begin{alltt}
Template.news.events(\{
    'submit .add-news'(event) \{
        // предотвращаем обработку формы по умолчанию 
        // (т.к. работаем на стороне клиента)
        event.preventDefault();
		
        // получаем данные из формы
        const target = event.target;
        const title = target.title.value;
        const text = target.text.value;
		
        // вставляем новый документ в коллекцию
        News.insert(\{
            title: title,
            text: text,
            date: new Date(), 
            username: Meteor.user().username
        \}); 
		
        // очищаем форму для добавления новых в будущем
        target.title.value = '';
        target.text.value = '';
    \}
\});
\end{alltt}

Аналогичным образом можно обрабатывать другие HTML-события.

\subsection{Прочее}
Инициализацию приложения можно производить с помощью файла
``imports/startup/server/fixtures.js''. В нашем приложении
в этом файле происходит инициализация хранилища тестовыми данными
(один пользователь и несколько новостей), если хранилище
пустое.

Для подключения файлов ресурсов (например, стилей) нужно
создать каталог ``public'' в корне приложения. Его
подкаталоги будут доступны по абсолютному пути,
например ``/images/logo.png''.

\subsection{Дальнейшая работа}
При разработке рассматриваемого новостного приложения не были
рассмотрены следующие важные вопросы.

Для обеспечения безопасности доступа к данным необходимо
запретить редактирование данных на стороне клиента.
Для этого необходимо удалить пакет ``insecure''
и, желательно, переопределить методы insert и update
коллекций.

Данные могут распространяться по модели побписчик/издатель.
Так, по умолчанию, meteor будет обновлять данные
на стороне клиента автоматически по мере поступления.
Для обеспечения безопасности доступа к данным, 
которые могут быть приватными, нужно удалить пакет 
``autopublish'' и переопределить методы
``Meteor.publish'' и ``Meteor.subscribe'' для
каждой из коллекций.

Также, для обеспечения безопасности,
в Meteor доступен роутинг и рендеринг 
на стороне сервера.

Кроме того, в Meteor доступно облегченное использование препроцессоров стилей,
в частности Less.

\section*{Заключение}
Программный каркас Meteor является мощным средством для создания
web-приложений реального времени на всех уровнях приложения на одном
и том же языке (JavaScript). 

В ходе разработки приложения-примера для создания новостей у автора
сложились следующие личные субъективные впечатления о Meteor:

\begin{itemize}
	\item Meteor облегчает первоначальную настройку и развертываение приложения,
	так как в его дистрибутиве распространяются все необходимые средства. Однако,
	``advanced''-настройка приложения, например для взаимодействия с MongoDB,
	не встроенной в Meteor, может быть затруднена и скрыта под уровнями абстракций.
	\item То же самое касается и использования стороннего NodeJS.
	\item Документация фреймфорка кажется полной, однако оформлена она не самым лучшим
	образом. Так, в ней есть раздел ``Tutorial'' в котором происходит разработка
	``простейшего'' приложения, которое на поверку оказывается значительно более
	сложным, чем аналогичные во многих других фреймворках. Также, из текста данного 
	``Tutorial'' не всегда бывают понятны некоторые моменты или не видна
	полная картина на какой0либо момент. Автору приходилось активно обращатся 
	к репозиторию с полным примерным приложением на GitHub (см. \cite{todos}).
	\item Сомнительными кажутся решения по обеспечению безопасности.
	Так, автору сложно представить приложение, в котором допустимо оставлять
	свободный доступ к редактированию базы данных на стороне клиента.
	\item Тем не менее, фреймфорк кажется достаточно удобны для создания 
	одностраничных приложений реального времени, в частности благодаря
	удачной модели разбиения на файлы. 
\end{itemize}

Полные исходники Новостного приложения могут быть найдены на GitHub \cite{news-app}.

\begin{thebibliography}{99}
\bibitem{meteor} Страница фреймфорка Meteor:
https://www.meteor.com/	

\bibitem{atmosphere} Страница менеджера пакетов Atmosphere:
https://atmospherejs.com/

\bibitem{mongo} Страница СУБД MongoDB:
https://www.mongodb.com/

\bibitem{tutorial} Официальный Tutorial Meteor:
https://www.meteor.com/tutorials

\bibitem{tutorial} Официальный Guide Meteor:
https://guide.meteor.com/
	
\bibitem{todos} Полные исходники приложения из Tutorial фреймворка Meteor:
https://github.com/meteor/todos	
	
\bibitem{news-app}	Исходный код новостного приложения: 
https://github.com/antonandreyev/meteor-example
	
\bibitem{inside} Статья ``Как устроен Meteor изнутри'':
https://habrahabr.ru/post/205878/	
	
\end{thebibliography}

\end{document}
